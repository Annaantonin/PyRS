\documentclass[12pt,bezier,amstex]{report}  % include bezier curves
\renewcommand\baselinestretch{1.0}           % single space
%\pagestyle{empty}                            % no headers and page numbers
\oddsidemargin -10 true pt      % Left margin on odd-numbered pages.
\evensidemargin 10 true pt      % Left margin on even-numbered pages.
\marginparwidth 0.75 true in    % Width of marginal notes.
\oddsidemargin  0 true in       % Note that \oddsidemargin=\evensidemargin
\evensidemargin 0 true in
\topmargin 0.25 true in        % Nominal distance from top of page to top of
\textheight 9.0 true in         % Height of text (including footnotes and figures)
\textwidth 6.375 true in        % Width of text line.
\parindent=0pt                  % Do not indent paragraphs
\parskip=0.15 true in
\usepackage{color}              % Need the color package
\usepackage{epsfig}

\usepackage{algorithmic}


\title{NRSF2 Design Document Appendix: Strain and Stress Calculation}

\begin{document}

\maketitle

{\bf Strain}, aka, unconstrained strain, is measured as the fraction change from a reference state ($d_0$).
\begin{eqnarray}
\epsilon_{ij} &=& \frac{d_{ij} - d_0}{d_0}
\end{eqnarray}

{\bf Residual stress} is determined by measuring stress along {\bf\it 3} orthogonal directions
\begin{eqnarray}
\sigma_{ij} &=&
	\frac{E}{(1 + \nu)}[\epsilon_{ij} + \frac{\nu}{1-2\nu}(\epsilon_{11} + \epsilon_{22} + \epsilon_{33})]
\end{eqnarray}

where 
\begin{itemize}
\item $\nu$ is {\it Poisson's Ratio}.
\item $E$ is {\it Young's Modulus}.
\item $\epsilon_{ij}$ are strains.  Be noted that 
	\begin{itemize}
	\item  $\epsilon_{ij}$ with $i = j$ are principle strains.  But not all all three orthogonal strains are equivalent to principle strains.
	\item The off-diagonal strain component, i.e., $\epsilon_{ij}$ with $i\neq j$ are all set to {\bf zero}.  It is very hard to measure these values in HB2B's setup.
	\end{itemize}
\end{itemize}

In {\bf plane strain}, $\epsilon_{33}$ is {\bf zero}.

In {\bf plane stress}, $\sigma_{33}$ is {\bf zero}.  
Therefore, $\epsilon_{33}$ can be calculated from $\epsilon_{11}$ and $\epsilon_{22}$ from $\sigma_{33} = 0$. 
\begin{eqnarray}
\sigma_{33}
	&=&	0	\\
	&=& \frac{E}{(1 + \nu)}[\epsilon_{33} + \frac{\nu}{1-2\nu}(\epsilon_{11} + \epsilon_{22} + \epsilon_{33})]	\\
\epsilon_{33}
	&=& - \frac{\nu}{(1-2\nu)} (\epsilon_{11} + \epsilon_{22}) + - \frac{\nu}{(1-2\nu)} \epsilon_{33} \\
\epsilon_{33} 
	&=& \frac{\nu}{\nu-1}(\epsilon_{11} + \epsilon_{22})
\end{eqnarray}

Therefore, for both plain stress and plain strain, $d$ from 2 principle directions are enough, but not 3.

\end{document}

